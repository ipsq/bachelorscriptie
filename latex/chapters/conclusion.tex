The goal of this paper was to replicate the findings of \textcite{swanson_big_2017} and to extend their research by including an additional dimension reduction technique in the form of sparse principal components. 
The model forecasts differ in predictive power very significantly based on the economic situation of the sample that was used to estimate the model. 
Simple models appear to perform better in forecast scenarios when both an economic recession and upturn are included in the sample.
We believe this to be the affect of sudden changes in interest rates by the Federal Reserve.
These changes cannot be captured accurately by our models and introduce errors.
It is also apparent that when a sample only includes an economic upturn, as was the case with our second sample, the dynamic Nelson-Siegel models perform best in predictive accuracy. 
When comparing different DNS model formulations, there was no clear winner for all maturities. 
We find that VAR formulations perform better for smaller maturities and AR formulations perform better for longer maturities. 
For diffusion index models, we cannot draw any conclusions from our results. 
It appeared there is no significant difference between factor augmented and non factor augmented models. 
For our extension, it is clear that sparse principal component analysis is outperforming standard principal component analysis significantly.
This leads us to believe that the increase in parsimony of the underlying factors manages to extract more important information out of the data than standard principal component analysis.