\section{Yield data}
For our yield data, we will be making use of the dataset published by~\textcite{Grkaynak2007}. 
This dataset consists of United States Department of Treasury zero-coupon, continuously compounded, yield curve data. 
The dataset contains daily yield curve data from the 14th of June 1961 to the 1st of May 2020, at the time of writing. 
This dataset contains maturities from 1 year up to and including 30 years. 
For our research only the maturities from 1 year up to and including 10 years are used. 
Because yield curve estimates only extended out to 7 years until the 16th of August 1971~\parencite[see][p.~19]{Grkaynak2007}, our sample will start from August 1971 to April 2020. 

\section{Macroeconomic data}
For our macroeconomic data, used for constructing the (macro-)factors, we will be making use of the FRED-MD dataset published by~\textcite{McCracken2016}. 
This dataset is developed by the United States Federal Reserve Bank of St. Louis and consists of 134 monthly U.S. macroeconomic variables from January 1959 to March 2020, at the time of writing. 
For our research, only 105 variables are used.

\section{Samples}
For our research, we use the following samples to compare our various different models:
\begin{enumerate}
	\item Subsample 1, 1992-01-01 to 1999-12-01
	\item Subsample 2, 2000-01-01 to 2007-12-01
	\item Subsample 3, 2008-01-01 to 2016-12-01
	\item Full sample, 1992-01-01 to 2016-12-01
\end{enumerate}