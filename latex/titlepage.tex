\begin{titlepage}	
	\begin{center}
		\textsc{\Large{Erasmus University Rotterdam}} \\[.7cm]
		\textsc{Erasmus School of Economics} \\[0.5cm]
		
		\rule{\linewidth}{0.5mm} \\[0.4cm]
		\huge{\bfseries{Comparing small data models versus big data models for forecasting interest rates\footnote{The views stated in this thesis are those of the author and not necessarily those of the supervisor, second assessor, Erasmus School of Economics or Erasmus University Rotterdam.}}} \\
		\rule{\linewidth}{0.5mm} \\[.5cm]
		
		\textsc{\large{Bachelor Thesis Econometrics and Operations Research}} \\[.5cm]
		
		\large{Martijn Riemers (445591)} \\
		\large{\href{mailto:martijnriemers@student.eur.nl}{martijnriemers@student.eur.nl}} \\[.75cm]	
		
		\begin{minipage}[t]{0.4\textwidth}
		\center
		\large{\emph{Supervisor:}}\\
		\large{T. van der Zwan}
		\end{minipage}
		%
		\begin{minipage}[t]{0.4\textwidth}
		\center
		\large{\emph{Second assessor:}} \\
		\large{Prof. dr. D.J.C. van Dijk}
		\end{minipage}\\[.75cm]
		
		\large{\emph{Date final version:}} \\
		\large{\today} \\[4cm]
	\end{center}
		
		
	{\setlength{\parindent}{0cm}
	\textbf{Abstract} \\
	In this paper, we model the yield curve of government bonds using a wide range of different small and big data models. 
	We then compare their predictive accuracy against a benchmark AR(1) model to find the best performing forecasting model. 
	We include factor augmented models with standard principal component analysis and sparse principal component analysis and compare the respective models their predictive accuracy. 
	We find that models predictive accuracy largely depends on the economic situation. 
	We also find that models with factors estimated with sparse principal component analysis perform better than factors estimated with standard principal component analysis. 
	}
\end{titlepage}